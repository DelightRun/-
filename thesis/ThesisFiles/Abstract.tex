% 中文摘要
\begin{abstract}
随着近几年中国汽车保有量的持续增加,智能交通系统以及自动驾驶技术俨然已成为十分
热门的研究领域。而机动车车牌识别技术作为这两项技术的基础和关键,有着十分重要的研
究意义和应用价值。目前国内外许多专家学者都针对车牌识别产开了深入的研究,并涌现出
许多新技术、新方法。此外,随着近几年机器学习技术,尤其是基于神经网络的深度学习技
术的发展,许多相关技术被广泛应用于计算机视觉任务中并取得了相当出色的效果。本
文主要对车牌识别系统所涉及的车牌检测、车牌定位、车牌字符分割和车牌识别四个子系统
展开研究,并运用最新的深度学习技术实现一个简单的车牌识别系统。全文工作将分为以
下几点:
第一,本文首先阐述本课题的研究背景及意义,并以此说明研究和实现车牌识别系统的必要
性;
第二,本文对车牌检测方法展开研究。车牌定位是整个系统的第一步,直接决定了后续步
骤是否能正确进行。传统的车牌检测方法有:基于边缘检测的方法、基于形态学的方法、基
于颜色划分的方法和基于纹理特征的方法。然而这些方法由于过度依赖人工设定的特征和规
则,有着不够鲁棒、缺乏语义信息等缺点。Region CNN(简称RCNN)技术是近几年新出现的
一种新的目标检测算法,并在诸多数据集的测试中取得了目前最优的效果。本文将研究如何通
过RCNN技术进行车牌检测;
\end{abstract}
第三,本文尝试使用CNN回归车牌顶点坐标的方法以实现车牌定位,来克服传统方法严重
依赖手工特征和手工规则、鲁棒性差等缺点。
第四,本文对车牌字符分割方法展开研究。车牌字符分割问题可以看做对车牌图片中的文字进行
提取的问题。关于自然场景中的文字提取,已有大量相关研究,其中基于Extremal Region
(简称ER)的方法取得了非常理想的效果。基于ER的方法可分为Maximum Stable
Extremal Region(简称MSER)和Class-specific ER(简称CSER)两种。本文
对比两种方法,发现CSER方法可以克服MSER方法许多缺点,有着更理想的效果,并采用该方
法进行车牌字符分割。
第五,在图像识别领域,目前最理想的方法当属卷积神经网络(简称CNN),它有着更好的
识别准确率和更强的鲁棒性。因此本系统采用CNN的方法进行车牌识别。
第六,通过对以上技术的研究,本文实现了一个简单的车牌识别系统,在本系统中,使用
RCNN进行车牌检测;使用CNN回归的方法进行车牌定位;使用CSER的方法进行车牌字符分割;
最后使用CNN技术进行车牌识别。
\keywords{车牌检测,车牌定位,车牌字符分割,车牌识别,深度学习}

% 英文摘要
\begin{enabstract}
With the continued increase in vehicle ownership in China in recent years,
intelligent transportation system and autopilot technology has became a very
popular field of research. As the foundation and key of these two technologies,
the vehicle license plate recognition technology has great significance in both
research and application. Many experts and scholars have started the in-depth
research on vehicle license plate recognition, and many new technologies and
methods have be introduced. In addition, with the development of machine
learning technology, especially deep learning technology which is based on
neural network, many of the relevant technologies is widely used in computer
vision tasks and have achieved state-of-art results. This paper focuses on four
essential subsystem in the vehicle license plate recognition system involved
detection, location, segmentation and recognition. We implement a simple vehicle
license plate recognition systems based on the up to date deep learning
technologies. We summarize the whole wore as follows:
First, we illustrates the certain necessity of the research and implementation
of this vehicle license plate recognition system by describing the background and
significance of the research.
Second, this paper researches vehicle license plate detection methods.
License plate detection is the first step of the entire system, it will
determine whether the next steps can execute correctly. Traditional methods of
license plate detection inclueds: Edge-based approach, morphology-based
approach, color-based approach and projection-based approach. But these methods
are highly dependent on hand-made features and rules, so they all has some
disadvantages such as poor robustness and lack of context information. In recent
years, a new approach called Region CNN(RCNN) is proposed for object detection
task, and get state-of-art results on many test datasets. This paper will
researches the approach of vehicle license plate detection based on RCNN
technology.
Third, in this paper, we try to location vehicle license plate by regressing the
vertex positions using CNN, in order to overcome the shortcomings of traditional
methods such as the dependency of hand-made features and features as well as the
poor robustness.
Fourth, this paper studies some license plate segmentation methos. The vehicle
license plate segmentation problem can be view as the problem of detecting
characters from license plate pictures. There are lots of researches on text
extraction from natural scence, and a popular approach called Extremal Region(ER)
get the state-of-art result on this task. The methods based on ER can be mainly
divided into two kind: Maximum Stable Extremal Region(MSER) and Class-specific
ER(CSER). In this paper, we test on both of these two methods, and found that
the CSER method can overcome many shortcomings of MSER and has a better result.
So we use the CSER way in our final system.
Fifth, in the field of image recognition, there is no doubt that CNN is the
state-of-art approach currently, it has a higher accuracy and a better
robustness. So we adapt the CNN for license plate recognition in our final
system.
Sixth, through the study of the technologies above, this paper implements a
simple license plate recognition system. In our recognition system, we use RCNN
for detection, CNN regression for location, CSER for character segmentation and
CNN for the final recognition.\cite{Ren:2015ug}
\end{enabstract}
\enkeywords{vehicle license plate detection, vehicle license plate location,
  vehicle license character segmentation, vehicle license plate recognition,
  deep learning}